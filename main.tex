\documentclass[12pt]{article}
\usepackage{sbc-template}
\usepackage{graphicx,url}
\usepackage[brazil]{babel} 
\usepackage[utf8]{inputenc}
\usepackage{placeins}
\usepackage{longtable}
\usepackage{multirow}

\sloppy

\title{Plano de gerenciamento de riscos:\\Site de venda de ingressos}

\author{
    \begin{minipage}{\textwidth}
        Carlos Alberto\inst{1},
        Daniela Menezes\inst{2},
        Danilo Caldeira\inst{3},
        Gabriel Augusto R. dos Reis\inst{4},
        Giovani Araújo\inst{5},
        Guilherme Ferreira Faioli Lima\inst{6},
        Indianara Santos Rodrigues\inst{7},
        Luiz Carlos Silva Júnior\inst{8}
    \end{minipage}
}
\address{
    Departamento de Computação e Sistemas\\
    Universidade Federal de Ouro Preto (UFOP) -- João Monlevade, MG\\
    \\16.2.8394 \inst{1}, 
    16.2.8540 \inst{2},
    16.2.8512 \inst{3},
    16.2.8105 \inst{4},
    16.2.8149 \inst{5},
    16.1.8243 \inst{6},
    17.2.8246 \inst{7},
    15.1.5881 \inst{8}
}

\begin{document}

    \maketitle

    \begin{abstract}
        This management plan aims to identify all risks associated with the project, as well as understand them in order to avoid them and in case they occur to know how to correct the failure. The identification and evaluation of these risks will be carried out in a brainstorming, in addition to using SWOT analysis, detailed in figure \ref{fig:swot}, as well as checklists and lessons learned from previous projects.
    \end{abstract}

    \begin{resumo}
        Este plano de gerenciamento tem como objetivo identificar todos os riscos associados ao projeto, assim como entendê-los afim de evitá-los e caso eles ocorram saber como corrigir a falha. A identificação e avaliação desses riscos serão realizadas em um brainstorming, além de utilizar a análise SWOT, detalhada na figura \ref{fig:swot}, assim como checklists e as lições aprendidas com projetos anteriores.
    \end{resumo}
    
\tableofcontents
\newpage

    \begin{figure}[ht]
        \centering
        \includegraphics[scale=0.3]{./Images/swot.png}
        \caption{Gráfico SWOT}
        \label{fig:swot}
    \end{figure}

    \section{Identificar riscos associados ao projeto}
        \begin{itemize}
            \item Perda de dados.
            \item Layout ruim.
            \item Usabilidade confusa.
            \item Repostas erradas do Banco de dados.
            \item Erros de sincronização com os métodos de pagamento.
            \item Erros associados ao mecanismo de busca.
            \item Erros com editor de conteúdo.
            \item Erros relacionados a portabilidade.
            \item Número de usuários maior que o planejado.
            \item Usuários resistentes ao sistema.
            \item Ataque Hacker.
            \item Marketing ruim do site.
            \item Risco financeiro.
            \item Falta de atualização dos cinemas ou sessões.
            \item Preços não competitivos no mercado ou com o próprio cinema localmente.
            \item Satisfação do cliente. 
            \item Inutilização dos cinemas.
        \end{itemize} 
        
    \section{Avaliar probabilidade de ocorrência}
        De acordo com a Figura \ref{fig:swot}, os riscos identificados serão qualificados de acordo com sua probabilidade de ocorrência.
        
        \begin{itemize}
            \item  Probabilidade:
            \begin{itemize}
                \item Baixa - A probabilidade de ocorrência do risco pode ser considerada pequena.               
                \item Média - Existe uma probabilidade razoável de ocorrer o risco.
                \item Alta - O risco é iminente.
            \end{itemize}
            \item Gravidade:
            \begin{itemize}
                \item Baixa - O impacto do evento de risco é irrelevante, tanto em termos de custo, prazo, podendo ser resolvido facilmente.
                \item Média - O impacto de evento de risco é relevante e necessita de gerenciamento mais preciso.
                \item Alta - O impacto do evento de risco é elevado e, caso não haja uma interferência imediata e precisa, os resultados serão comprometidos.
            \end{itemize}
        \end{itemize}
       
        \begin{table}[h]
            \centering
            \label{table:gravidade_e_riscos}
            \resizebox{\textwidth}{!}{
                \begin{tabular}{|l|c|c|}
                    \hline
                    \multicolumn{1}{|c|}{\textbf{Riscos}} & 
                    \multicolumn{1}{|c|}{\textbf{Probabilidade}} & 
                    \multicolumn{1}{|c|}{\textbf{Gravidade}} \\ \hline
                    Perda de dados & baixa & alta \\ \hline
                    Layout ruim & baixa & baixa \\ \hline
                    Usabilidade confusa & baixa & média \\ \hline
                    Repostas erradas do Banco de dados & baixa & alta \\ \hline
                    Métodos de pagamento & baixa & alta \\ \hline
                    Mecanismo de busca & média & baixa \\ \hline
                    Portabilidade & média & alta \\ \hline
                    Número de usuários maior que o planejado & baixa & baixa \\ \hline
                    Ataque Hacker & média & alta \\ \hline
                    Marketing ruim do site & média & baixa \\ \hline
                    Risco financeiro & baixa & alta \\ \hline
                    Falta de atualização dos cinemas ou sessões & alta & alta \\ \hline
                    Preços não competitivos no mercado ou com o próprio cinema localmente & alta & alta \\ \hline
                    Satisfação do cliente & baixa & média \\ \hline
                    Inutilização dos cinemas & baixa & alta \\ \hline
                \end{tabular}
            }
            \caption{Probabilidade e gravidade dos riscos}
        \end{table}
    \FloatBarrier

    \section{Estimar impacto}

        \begin{center}
            \begin{longtable}{|p{2.0cm}|p{3.2cm}|c|p{3.2cm}|c|}
                
            
                \hline 
                \multicolumn{1}{|c|}{\textbf{Riscos}} & 
                \multicolumn{1}{c|}{\textbf{Descrição}} & 
                \multicolumn{1}{c|}{\textbf{Probabilidade}} &
                \multicolumn{1}{c|}{\textbf{Impacto}} &
                \multicolumn{1}{c|}{\textbf{Gravidade}} \\ \hline 
                \endfirsthead

                \multicolumn{5}{c}%
                {{\bfseries \tablename\ \thetable{} -- continuação da última página}} \\
                \hline 
                \multicolumn{1}{|c|}{\textbf{Riscos}} & 
                \multicolumn{1}{c|}{\textbf{Descrição}} & 
                \multicolumn{1}{c|}{\textbf{Probabilidade}} &
                \multicolumn{1}{c|}{\textbf{Impacto}} &
                \multicolumn{1}{c|}{\textbf{Gravidade}}\\ 
                \endhead

                \hline 
                \multicolumn{5}{|r|}{{Continua na próxima página}} \\ \hline
                \endfoot

                \hline \hline
                \endlastfoot
                
                Perda de dados & Devido a problemas no banco de dados, os dados podem ser momentaneamente perdidos & baixa & O software pode ficar inoperante até que o problema seja resolvido. & alta \\ \hline
                
                Layout ruim & Caso o site não funcione corretamente não irá proporcionar uma experiência agradável ao usuário  & baixa  & Perda de usuários e consequentemente perda de lucro & baixa  \\ \hline
                
                Usabilidade confusa & Uma navegação complexa pode fazer com que o visitante abandone o site. & baixa  & Perda de visitantes & média \\ \hline
                
                Repostas erradas do Banco de dados & Banco de dados retornam ingressos ou dados relativos ao ingressos ou sessões de forma errada & baixa & Gera compras erradas, inesperadas ou inexistentes, causando transtorno aos clientes, prejuízo ao site e má reputação perante os clientes & alta  \\ \hline
                
                Métodos de pagamento & Método de pagamento para de responder como o esperado, deixando de efetuar ou confirmar as transações & baixa & Parada nas vendas, gerando prejuízos até que o sistema retorne ao normal & alta  \\ \hline
                
                Mecanismo de busca & Campos de busca são áreas bastante utilizadas para encontrar o conteúdo do site o mais rápido possível. & média  & Visitante frustrado e consequentemente perda de visitantes  & baixa  \\ \hline
                
                Portabilidade & Site funciona de forma inesperada em browsers ou dispositivos diferentes & média & Usabilidade ruim do site & alta \\ \hline
                
                Número de usuários maior que o planejado & Número de usuários acessando simultaneamente o site é maior que o esperado & baixa & Clientes ativos não terão problemas, a medida que os novos visitantes não conseguirão acessar até que um antigo saia e libere o recurso & baixa \\ \hline
                
                Ataque Hacker & Devido a uma falha de segurança em qualquer parte da infraestrutura, uma pessoa mal-intencionada, pode invadir o servidor e ter acesso a dados privados, podendo alterá-los ou divulgá-los de forma indevida. & média & A empresa pode sofrer processos legais caso os dados privados sejam descobertos ou o site pode ficar temporariamente inoperante. & alta \\ \hline 
                
                Marketing ruim do site & divulgação de forma errada, sem alcance ou desinteressante aos possíveis clientes & média & Desinteresse de novos clientes & baixa  \\ \hline
                
                Risco financeiro & Incerteza sobre futuro financeiro da empresa & baixa & Revisão nos planos da empresa & alta  \\ \hline 
                
                Falta de atualização dos cinemas ou sessões & Não atualizar os dados dos cinemas ou sessões de filmes & alta & Compras erradas ou falta de opções de compra & alta  \\ \hline 
                
                Preços não competitivos no mercado ou com o próprio cinema localmente & Preços mais caros ou não vantajosos quando comparado com concorrentes ou o próprio cinema & alta & Não serão realizadas novas vendas & alta  \\ \hline 
                
                Satisfação do cliente & reclamações sobre o site, principalmente nas redes sociais & baixa & danos à identidade digital & média  \\ \hline 
                
                Inutilização dos cinemas & Tecnologia ficar ultrapassada ou ser substituída & baixa & Adequação ao novo mercado ou tecnologia & alta  \\ \hline 
              
                \caption{Descrição dos riscos e seus Impactos}
                \label{Descricao_E_Impacto_Dos_Riscos}
            \end{longtable}
        \end{center}

        
    \subsection{Estabelecer plano de contingência}
       
       \begin{itemize}
       \item Perda de dados: A melhor forma de evitar a perda de dados é fazer backups periódicos ou sistemas distribuídos.
       
       \item  Layout ruim: Para que o site funcione corretamente e proporcione uma experiência agradável ao usuário é necessário que a parte visual/estética estejam alinhados. Caso haja exagero nas cores, tamanho de imagem inadequado que atrapalhe a qualidade visual da tela é necessário que o layout seja refeito.
       
        \item Usabilidade confusa: É necessário que parte do site seja refeito de maneira simples, inidentificável e direta, seguindo o padrão na construção das páginas, agrupar o conteúdo do site em categorias lógicas fazendo com que a resposta por uma procura seja mais direta possível.
        
        \item Respostas erradas do banco de dados: Caso seja identificado um método de pequisa que gere uma resposta errada, um desenvolvedor de manutenção do software deve ser chamado para que posso ser analisado o motivo do erro e então liberada uma versão de correção do software o mais rápido possível.
        
        \item Métodos de pagamento: Contactar imediatamente a empresa terceirizada responsável pelo pagamento para que seja reparado ou se necessário, efetuar a troca da empresa terceirizada.
       
        \item Mecanismo de busca: Caso as buscas gerem resultados inesperados, as consultas feitas no banco de dados devem ser repensadas para que os erros não se repitam.
        
        \item Portabilidade: Caso seja identificado um dispositivo ou browser que gere um layout     método de pequisa que gere um layout "quebrado", deve-se verificar quais as características do dispositivo para que possa ser corrigido e então liberar uma versão de correção do software o mais rápido possível.
        
        \item Número de usuários maior  que o planejado: Caso o site não suporte a quantidade de usuários, será necessário aumentar a capacidade do servidor.
        
        \item  Ataque Hacker: Verificar em qual área o ataque foi feito, caso sejam dados privados, alertar a todos os envolvidos imediatamente, caso dados sejam modificados, o backup deverá ser consultado e restaurado, caso contrário reestabelecer o servidor para que o site volte a operar normalmente.
        
        \item Marketing ruim do site: Após verificar a efetividade do marketing atual (cruto, médio e longo prazo), refazer todo o conceito de divulgação do site.
        
        \item Risco financeiro:  Revisão constante das finanças e planejamento financeiro futuro detalhado e discutido com os administradores da empresa.
        
        \item Falta de atualização dos cinemas ou sessões: contratar novo funcionário para tal função ou integrar banco de dados do site com o do cinema.
        
        \item Preços não competitivos no mercado ou com o próprio cinema localmente: Revisão na metodologia financeira da empresa, afim de diminuir os preços.
        
        \item Satisfação do cliente: Para evitar esse risco, um SAC (Serviço de Atendimento ao Consumidor) precisa ser colocado em prática na estratégia de marketing digital. Reclamações de qualquer natureza devem ser respondidas.
        
         \item Inutilização dos cinemas: Adequação a nova tecnologia existente e/ou mercado.
        
       \end{itemize}

    \section{Riscos de projeto}
        Os riscos de projeto estão relacionados com aspectos operacionais, organizacionais e contratuais, que são de responsabilidade do Gerente do Projeto. Em sua descrição está incluído o relacionamento com os fornecedores, restrições contratuais, interfaces externas, requisitos, clientes, organização e limitações de recursos a serem utilizados. Tais riscos ameaçam diretamente o plano do projeto, podendo vir a atrasar o cronograma e também elevar os custos do mesmo, uma vez que o maior risco dos projetos de software é o financeiro pois, a obtenção de recursos orçamentários pode variar de acordo com o mercado financeiro, por exemplo.
    \section{Riscos técnicos}
        Consiste em variáveis que podem guiar o sucesso ou insucesso do projeto em si. Tem como característica principal definir a probabilidade em que eventos indesejados aconteçam durante o projeto de desenvolvimento de software. Identifica potenciais problemas que podem vir a ocorrer, como implementação, interface, manutenção e mais. Novas tecnologias também é um fator de risco muito comum, pois com o avanço da tecnologia a cada dia é lançado algo novo que pode impactar no projeto.
    \section{Riscos de negócio}
        Riscos de negócio ameaçam a viabilidade do projeto. Podemos observar alguns pontos como mudança da equipe que está relacionada com o trabalho e a falta de orçamento que será colocado sobre o projeto. Outro risco seria um erro de projeção sobre o produto, fazendo com que esse, depois de terminado, fosse notado que os clientes não queriam, ou não precisavam desse produto. 
        Dentro dos riscos de negócio podemos caracterizar o risco gerencial que seria a perda de apoio da gerencia empresa devido a mudança de foco da mesma. 
        Sobre os riscos acima citados podemos perceber que o nosso projeto em si tem pouca probabilidade de ser afetado, tendo em mente que o mercado de cinema está em crescente e temos uma equipe consolidada para realizar o projeto.
        Todavia, os riscos de negócio, por estarem relacionados diretamente com o desenvolvimento do projeto, seus resultados dependem também dos riscos orçamentários e de marketing, estes estão citados a seguir.
    \section{Risco de marketing}
        O marketing é uma grande aliado para empresa, pois ele otimiza o lucro da empresa através da oferta de mercadorias ou serviços de acordo com a necessidade e preferência dos consumidores. 
        O marketing pode ser realizado através de propagandas, pesquisas de mercado, design, entre outras. 
        Com o objetivo de saber como anda a questão do marketing de vendas de ingressos de cinema, fizemos pesquisa com algumas pessoas para saber se elas têm acesso a publicidade envolvendo a venda de ingressos. Bom, a pesquisa tão teve um resultado bom como podemos ver na tabela\ref{table:freq_divulgacao}.
        
      \begin{table}[h]
            \centering
            \resizebox{\textwidth}{!}{
                \begin{tabular}{|c|c|c|c|c|c|c|c|c|c|c|c|}
                    \hline
                    \multicolumn{1}{|c|}{\textbf{}} & 
                    \multicolumn{1}{c|}{\textbf{Carlos}} & 
                    \multicolumn{1}{c|}{\textbf{Danilo}} & 
                    \multicolumn{1}{c|}{\textbf{Gabriel}} & 
                    \multicolumn{1}{c|}{\textbf{Giovani}} & 
                    \multicolumn{1}{c|}{\textbf{Giovanna}} & 
                    \multicolumn{1}{c|}{\textbf{Guilherme}} & 
                    \multicolumn{1}{c|}{\textbf{Indianara}} &
                    \multicolumn{1}{c|}{\textbf{Luiz}} &
                    \multicolumn{1}{c|}{\textbf{Micael}} &
                    \multicolumn{1}{c|}{\textbf{Nicole}} &
                    \multicolumn{1}{c|}{\textbf{Vanessa}} \\ \hline
                    Frequentemente & & & & & & & & & & & \\ \hline
                    Raramente & X & X & X & X & X & X & X & X & & X & X \\ \hline
                    Nunca & & & & & & & & & X & & \\ \hline
                \end{tabular}
            }
            \caption{Com que frequência você vê divulgações de vendas de ingressos?}
            \label{table:freq_divulgacao}
        \end{table}
        \FloatBarrier
        
    É preocupante nos dias de hoje que não haja propaganda frequente sobre um determinado produto, visto que há grande procura de informações na internet sobre cinema e venda de ingressos visto na \ref{table:Meio_de_obter_info}, pois é uma maneira ágil e prática para a maiorias das pessoas. Se houvesse mais divulgação talvez não seria tão necessária a busca do cliente nos sites, mas sim já apareceria aquilo que o interessa fazendo que ele compre sempre e não só quando lança algum filme imperdível.
    
    \begin{table}[h]
            \centering
            \resizebox{\textwidth}{!}{
                \begin{tabular}{|c|c|c|c|c|c|c|c|c|c|c|c|}
                    \hline
                    \multicolumn{1}{|c|}{\textbf{}} & 
                    \multicolumn{1}{c|}{\textbf{Carlos}} & 
                    \multicolumn{1}{c|}{\textbf{Danilo}} & 
                    \multicolumn{1}{c|}{\textbf{Gabriel}} & 
                    \multicolumn{1}{c|}{\textbf{Giovani}} & 
                    \multicolumn{1}{c|}{\textbf{Giovanna}} & 
                    \multicolumn{1}{c|}{\textbf{Guilherme}} & 
                    \multicolumn{1}{c|}{\textbf{Indianara}} &
                    \multicolumn{1}{c|}{\textbf{Luiz}} &
                    \multicolumn{1}{c|}{\textbf{Micael}} &
                    \multicolumn{1}{c|}{\textbf{Nicole}} &
                    \multicolumn{1}{c|}{\textbf{Vanessa}} \\ \hline
                    Aplicativos & & & & & & & & & & & \\ \hline
                    Sites & X & X & X & X & X & X & X & X & X & X & X \\ \hline
                    Cartazes & & & & & & & & &  & & \\ \hline
                \end{tabular}
            }
            \caption{Na hora de escolher o filme para qual você irá comprar ingressos, qual meio você procura para obter informações?}
            \label{table:Meio_de_obter_info}
        \end{table}
        \FloatBarrier
    
    Um ponto positivo é que das pessoas que viram alguma publicidade disseram que essas publicidades tinham a ver com seu gostos.

      \begin{table}[h]
            \centering
            \resizebox{\textwidth}{!}{
                \begin{tabular}{|c|c|c|c|c|c|c|c|c|c|c|c|}
                    \hline
                    \multicolumn{1}{|c|}{\textbf{}} & 
                    \multicolumn{1}{c|}{\textbf{Carlos}} & 
                    \multicolumn{1}{c|}{\textbf{Danilo}} & 
                    \multicolumn{1}{c|}{\textbf{Gabriel}} & 
                    \multicolumn{1}{c|}{\textbf{Giovani}} & 
                    \multicolumn{1}{c|}{\textbf{Giovanna}} & 
                    \multicolumn{1}{c|}{\textbf{Guilherme}} & 
                    \multicolumn{1}{c|}{\textbf{Indianara}} &
                    \multicolumn{1}{c|}{\textbf{Luiz}} &
                    \multicolumn{1}{c|}{\textbf{Micael}} &
                    \multicolumn{1}{c|}{\textbf{Nicole}} &
                    \multicolumn{1}{c|}{\textbf{Vanessa}} \\ \hline
                    Sim & X & X & X & & & X & X & X & X & X & X \\ \hline
                    Não & & & & X & X & & & & & & \\ \hline
                \end{tabular}
            }
            \caption{Propagandas de vendas de ingressos geralmente estão relacionadas com seu gosto?}
            \label{table:Propagandas_ao_gosto}
        \end{table}
        \FloatBarrier
    Por fim, vimos que o marketing de vendas de ingressos de cinema é fraco entre as pessoas entrevistadas, e que com certeza o mercado poderia lucrar mais se houvesse mais divulgação. Por mais que a indústria do cinema se garanta e não entre em decadência, por sempre fazer as estreias do filme, o marketing responsável otimizar esse processo adquirindo sempre novos clientes e mantendo os que já tem.

    \section{Risco orçamentário}
        Vários riscos podem comprometer o andamento de um projeto, inclusive o orçamentário.
        Atraso no cronograma de desenvolvimento de um site de venda de ingresso pode causar custos maiores que o previsto. O fatores que geram esse prejuízo são, aumento de gastos com a empresa que desenvolve o software e enquanto mais demora a ficar pronto um site de vendas, maior é tempo que essa empresa não vende pela internet, sendo que nos dias de hoje muitos dos consumidores preferem a praticidade e agilidade de comprar pela internet.
        Manutenção, novas tecnologias também geram custos para empresa que deve estar sempre preparadas para essas situações.
        Depois que temos um site de vendas de ingresso de cinema pronto e funcionando corretamente é necessário alavancar as vendas através do marketing, mostrando para o público aquilo que ele quer ver de acordo com suas preferências, visto que em uma pesquisa feita a maioria das pessoas preferem assistir filme no cinema de acordo com a tabela \ref{table:preferencia_local_ver_filme}.
    
          \begin{table}[h]
            \centering
            \resizebox{\textwidth}{!}{
                \begin{tabular}{|c|c|c|c|c|c|c|c|c|c|c|c|}
                    \hline
                    \multicolumn{1}{|c|}{\textbf{}} & 
                    \multicolumn{1}{c|}{\textbf{Carlos}} & 
                    \multicolumn{1}{c|}{\textbf{Danilo}} & 
                    \multicolumn{1}{c|}{\textbf{Gabriel}} & 
                    \multicolumn{1}{c|}{\textbf{Giovani}} & 
                    \multicolumn{1}{c|}{\textbf{Giovanna}} & 
                    \multicolumn{1}{c|}{\textbf{Guilherme}} & 
                    \multicolumn{1}{c|}{\textbf{Indianara}} &
                    \multicolumn{1}{c|}{\textbf{Luiz}} &
                    \multicolumn{1}{c|}{\textbf{Micael}} &
                    \multicolumn{1}{c|}{\textbf{Nicole}} &
                    \multicolumn{1}{c|}{\textbf{Vanessa}} \\ \hline
                    Em casa & & X & & & & & X & & X & X & X \\ \hline
                    No cinema & & & X & X & X & X & & X & & & \\ \hline
                \end{tabular}
            }
            \caption{Preferência de local para ver filme}
            \label{table:preferencia_local_ver_filme}
        \end{table}
        \FloatBarrier
    
    Então o risco orçamentário pode ser visto de dois modos, como perda de dinheiro com gastos na hora de planejar um site e a perda de vendas que poderiam ser alcançadas através da publicidade.
    No fim de todo o processo do desenvolvimento do software é necessário que ele atenda todas as necessidades dos clientes e que esse site chegue ao maior número de pessoas possíveis.


    \section{Mitigação}
        A mitigação de risco envolve analisar os riscos que cada área da empresa está submetida, e os administrar, auxiliando equipe  de  projeto  no  desenvolvimento  de  uma  estratégia  para  lidar  com  o  risco.
        \subsection{Plano de mitigação}
        
        \begin{center}
        \begin{longtable}{|c|p{2.5cm}|p{4cm}|p{4cm}|}
        
            \hline 
                \multicolumn{1}{|c|}{\textbf{Área}} & 
                \multicolumn{1}{c|}{\textbf{Atividade}} & 
                \multicolumn{1}{c|}{\textbf{Risco}} & 
                \multicolumn{1}{c|}{\textbf{mitigação}} \\ \hline
            \endfirsthead

            \multicolumn{4}{c}%
                {{\bfseries \tablename\ \thetable{} -- continuação da última página}} \\
                \hline 
                \multicolumn{1}{|c|}{\textbf{Área}} & 
                \multicolumn{1}{c|}{\textbf{Atividade}} & 
                \multicolumn{1}{c|}{\textbf{Risco}} & 
                \multicolumn{1}{c|}{\textbf{mitigação}} \\ \hline
            \endhead

            \hline 
                \multicolumn{4}{|r|}{{Continua na próxima página}} \\ \hline
            \endfoot

            \hline \hline
            \endlastfoot
                
            \hline
                    
            Gerência Geral & Definição da duração do planejamento & erro nos tempos de duração das atividades e na quantidade de recursos a ser alocada para cada atividade & Consultar pessoas envolvidas em organizações de vendas de ingressos, montar cenários pessimistas e otimistas\\ \hline
            
            \multirow{2}{*}{Finanças} & Definição do preço do ingresso & Ter prejuízo; oferecer um serviço com qualidade abaixo do esperado & Considerar erro no cálculo do orçamento; aumentar esforço em patrocínio/ apoio; trabalhar sempre com determinada porcentagem de folga no orçamento \\ \cline{2-4} 
            
            & Definição do orçamento & Errar cálculos e previsão do orçamento; não atualizar dados com possíveis reajustes ou imprevistos. &  Elaborar mais de um orçamento com diversos cenários e atentar à todos os cálculos para que eles esgotem os principais gastos.\\ \hline
            
            \multirow{2}{*}{Infraestrutura} & Definição da infraestrutura de redes & quedas repentinas; O site não suportar o número de clientes esperado & Garantir que site esteja hospedado em uma boa empresa de hospedagem de sites e com um bom suporte técnico. \\ \cline{2-4} 
                     
            & Definição da segurança de acesso ao site & Vazamento de dados, invasões e fraudes; Site vulnerável a ataques & Investir em segurança da informação ou utilizar uma empresa de hospedagem que ofereça um grau alto de segurança.\\ \hline
            
            Marketing & Divulgação do filme e das vendas de ingresso & Desalinhamento na divulgação; não atingir a quantidade de pessoas desejadas; errar canais de comunicação & Criar um plano piloto de divulgação e pedir feedback aos cinemas; consultar pessoas experientes em divulgação de filmes e vendas de ingresso ou mesmo uma consultoria especializada.\\ \hline
            
            \multirow{2}{*}{Serviços gerais}& Contato com empresas de cinema e franquias & número de reuniões insuficientes. & Contratar uma empresa experiente e com capacidade de suprir qualquer variação de última hora. \\ \cline{2-4}
            
            & Definição do sistema de informação & Perda de informações. & Garantir que o SI é de fácil acesso a todos integrantes da equipe e que toda informação pode ser centralizada nele.\\ \hline
         
            \caption{Plano de mitigação}
            \label{table:Mitigacao}
        \end{longtable}
    \end{center}

        
        
    \section{Monitoramento}
            Monitoramento é uma atividade de rastreamento de projeto com três objetivos:
             \begin{itemize}
               \item Avaliar se riscos previstos de fato ocorrem;
               \item Garantir que passos de aversão de risco definidos para um risco estão sendo aplicados;
                \item Coletar informação que pode ser usada para análise de risco futura;
             \end{itemize}
        Problemas que ocorrem durante um projeto podem ser ligados a mais de um risco. Uma outra tarefa de monitoramento de riscos é alocação de origem.
        
        
        \begin{table}[h]
            \centering
            \resizebox{\textwidth}{!}{
                \begin{tabular}{|l|c|c|}
                    \hline
                    \multicolumn{1}{|c|}{\textbf{Riscos}} & 
                    \multicolumn{1}{|c|}{\textbf{Origem}} \\ \hline
                    Perda de dados & Banco de dados mal estruturado\\ \hline
                    Layout ruim & Falta de qualidade profissional\\ \hline
                    Usabilidade confusa & Falta de qualidade profissional \\ \hline
                    Repostas erradas do Banco de dados & Erros de usuário ou em programação com erros. \\ \hline
                    Métodos de pagamento & Erros de desenvolvimento, falta de qualidade do serviço, quando é terceirizado  \\ \hline
                    Mecanismo de busca & Erros de desenvolvimento de qualidade\\ \hline
                    Portabilidade & Falta de planejamento de requisitos \\ \hline
                    Número de usuários maior que o planejado & Falta de planejamento de requisitos\\ \hline
                    Ataque Hacker & Falhas de segurança \\ \hline
                    Marketing ruim do site & Falta de qualidade profissional \\ \hline
                    Risco financeiro & Falta de qualidade profissional em relação à administração \\ \hline
                    Falta de atualização dos cinemas ou sessões & Falta de qualidade do setor administrativo \\ \hline
                    Preços não competitivos no mercado ou com o próprio cinema localmente & Falta de qualidade profissional em relação à administração/financeiro \\ \hline
                    Satisfação do cliente & Falta de qualidade profissional  \\ \hline
                    Inutilização dos cinemas & Falta de qualidade profissional ou análise de mercado \\ \hline
                \end{tabular}
            }
            \caption{Riscos e possíveis causas}
            \label{table:riscos_e_causas}
        \end{table}
        \FloatBarrier
        
         \begin{table}[h]
            \centering
            \resizebox{\textwidth}{!}{
                \begin{tabular}{|l|c|c|}
                    \hline
                    \multicolumn{1}{|c|}{\textbf{Riscos futuros}} \\ \hline
                    Falta de suporte do sistema operacional. \\ \hline
                    Surgimento de tecnologias de grandes empresas como google que façam a mesma tarefa\\ \hline
                    Em um futuro um pouco distante, poderá haver a inexistência de cinemas, sendo substituídos por serviços como Netfix \\ \hline
                    Incompatibilidade com Frameworks atuais em uso\\ \hline
                \end{tabular}
            }
             \label{table:riscos_futuros}
            \caption{Riscos Futuros}
        \end{table}
        \FloatBarrier
        
        Através do monitoramento de TI é possível resolver alguns problemas enfrentados pela maioria dos gestores de TI, como:
        \begin{itemize}
               \item Falhas nos sistemas que imobilizam recursos;
               \item Dificuldade em medir indicadores de forma qualitativa;
                \item Sistemas que estão lentos sem que o gestor tenha o conhecimento do motivo;
                \item Usuários percebem os problemas antes da área de TI;
                \item Ou ainda, a TI não está alinhada ao negócio da empresa.
             \end{itemize}
        

    \section{Gerência de riscos}
        Consiste em avaliar e controlar os riscos que afetam o projeto, processo ou produto de software. A melhor maneira de descobrir os riscos é definir, inicialmente, os objetivos e metas do projeto. Os riscos são gerenciados tendo em vista objetivos específicos, podendo afetar apenas o trabalho que falta para alcançá-los. As perguntas importantes são: qual o risco contido no plano? Qual o risco contido no trabalho ainda restante? A incerteza é inerente a todas as suposições do projeto.	 
        
        Um dos conceitos fundamentais do Gerenciamento de Riscos é a perda. É preciso que haja um potencial de perda para que haja risco. A perda pode ter origem em um resultado negativo ou em uma oportunidade perdida. O resultado negativo pode ser, por exemplo, uma quantidade de erros inaceitável no produto, ou um atraso na data de entrega do mesmo. A oportunidade perdida pode se referir, por exemplo, a lucros perdidos, pela incapacidade de levar o produto ao mercado antes da 2 concorrência. 
        
        Outro conceito fundamental a ser considerado é o tempo. Embora o tempo seja um recurso valioso, não é possível acumulá-lo. Uma vez perdido, não há como recuperá-lo. Conforme o tempo passa, as opções viáveis vão se reduzindo. A perda do tempo é reduzida através do Gerenciamento de Riscos.

        A análise dos riscos é iniciada agrupando-se os riscos de mesma natureza, ou semelhantes. Devem ser determinados os fatores atuantes sobre os riscos, isto é, as variáveis que fazem a probabilidade de ocorrência ou o impacto (valor da perda) dos riscos flutuarem. Também devem ser determinadas as fontes de risco, ou seja, as respectivas causas, normalmente descobertas respondendo-se à pergunta “Por quê?” com relação a cada risco identificado. Em seguida, deve-se calcular a exposição referente a cada risco, definida como o produto da probabilidade de ocorrência do risco pelo respectivo impacto. A exposição é utilizada na priorização dos riscos. 

        O planejamento dos riscos inclui a definição de cenários para os riscos mais importantes, a definição de alternativas de solução para esses cenários, a escolha das alternativas mais adequadas, o desenvolvimento de um Plano de Ação de Riscos, assim como o estabelecimento de limiares ou disparadores para a ação. O acompanhamento dos riscos envolve a monitoração dos cenários de riscos, a verificação de que os limiares foram ou não atingidos, bem como a análise das medidas e indicadores referentes aos riscos. 

        A resolução dos riscos inclui a resposta aos eventos disparadores, a execução do Plano de Ação de Riscos, o acompanhamento da execução do plano e as eventuais correções de desvios.

        Uma ferramenta gratuita para o Gerenciamento de Riscos em geral (não apenas de software) é o software TRIMS, desenvolvido pelo BMP Center of Excellence, uma organização patrocinada pela Marinha e pelo Departamento do Comércio Norte Americanos, bem como pela Universidade de Maryland.

    \section{Conclusão}
    
        A partir das mudança decorrentes do cenário empresarial e econômico, as empresas necessitam de ferramentas que  demonstram a real situação socioeconômica da organização e do meio a qual está inserida, se basear em indicadores não garante o desemprenho esperado, qualquer simples erro de estratégia pode ocasionar mudanças bruscas e alterações nos balanços finais.
        Muitas vezes o investimento em um software está relacionado à busca por melhorias em processos, nem sempre esse é um fator determinante para a empresa se manter no mercado, ela pode continuar usando seus métodos já conhecidos, porém talvez não consiga aumentar os seus resultados com o passar do tempo.
        Mesmo com melhorias e investimentos, a elaboração e a execução de novos projetos podem apresentar riscos e incertezas, e esses fatores devem ser analisados. A ocorrência de riscos pode ameaçar o plano de projeto, tendo consequências podendo atrasar o cronograma e aumentar custos. Sendo assim, é preciso focar em planejamento.
        Afim de diminuir os possíveis incidentes é necessário que os riscos sejam identificados, analisados e classificados. Eles podem ser divididos de acordo com seu tipo, garantindo a qualidade do que está sendo produzido.
        A gestão de risco é uma ferramenta através da qual a empresa necessita, pois ela não retira a necessidade dos indicadores, mas se junta com eles e ao meio empresarial, por mais que os gestores tenham feito um bom plano de execução, sempre existirão riscos em um projeto de software.
        
    \nocite{*}
    \bibliographystyle{IEEEtran}
    \bibliography{sbc-template}
\end{document}